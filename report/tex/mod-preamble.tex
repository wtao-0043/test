%
% ---------------------------------------------------------------
% Copyright (C) 2012-2018 Gang Li
% ---------------------------------------------------------------
%
% This work is the default powerdot-tuliplab style test file and may be
% distributed and/or modified under the conditions of the LaTeX Project Public
% License, either version 1.3 of this license or (at your option) any later
% version. The latest version of this license is in
% http://www.latex-project.org/lppl.txt and version 1.3 or later is part of all
% distributions of LaTeX version 2003/12/01 or later.
%
% This work has the LPPL maintenance status "maintained".
%
% This Current Maintainer of this work is Gang Li.
%
%

\usepackage{cancel}
\usepackage{caption}
\usepackage{stackengine}
\usepackage{smartdiagram}
\usepackage{attrib}
\usepackage{amssymb}
\usepackage{amsmath} 
\usepackage{amsthm} 
\usepackage{mathtools}
\usepackage{rotating}
\usepackage{graphicx}
\usepackage{boxedminipage}
\usepackage{rotate}
\usepackage{calc}
\usepackage[absolute]{textpos}
\usepackage{psfrag,overpic}
\usepackage{fouriernc}
\usepackage{newtxmath}
\usepackage{pstricks,pst-3d,pst-grad,pstricks-add,pst-text,pst-node,pst-tree}
\usepackage{moreverb,epsfig,subfigure}
\usepackage{color}
\usepackage{booktabs}
\usepackage{etex}
\usepackage{breqn}
\usepackage{multirow}
\usepackage{natbib}
\usepackage{bibentry}
\usepackage{gitinfo2}
\usepackage{siunitx}
\usepackage{nicefrac}
%\usepackage{geometry}
%\geometry{verbose,letterpaper}
\usepackage{media9}
\usepackage{animate}
%\usepackage{movie15}
\usepackage{auto-pst-pdf}

\usepackage{breakurl}
\usepackage{fontawesome}
\usepackage{xcolor}
\usepackage{multicol}



\usepackage{verbatim}
\usepackage[utf8]{inputenc}
\usepackage{dtk-logos}
\usepackage{tikz}
\usepackage{adigraph}
%\usepackage{tkz-graph}
\usepackage{hyperref}
%\usepackage{ulem}
\usepackage{pgfplots}
\usepackage{verbatim}
\usepackage{fontawesome}


\usepackage[linesnumbered,lined,ruled]{algorithm2e}


\usepackage{todonotes}
\usepackage{tablefootnote}
\usepackage{threeparttable}
%\usepackage[numbers]{natbib}
\usepackage{bibentry}
\usepackage{bibunits}

\usepackage[utf8]{inputenc}
\usepackage{dtk-logos}
\usepackage{tikz}
\usetikzlibrary{arrows,chains,mindmap,shadows,automata,patterns,
    petri,shapes.geometric,shapes.misc, spy, trees,decorations.markings}


\usepackage{hyperref}
\hypersetup{ % TODO: PDF meta Data
    pdftitle={TULIP Lab},
    pdfauthor={Gang Li},
    pdfpagemode={FullScreen},
    pdfborder={0 0 0}
}


\usepackage{listings}
\lstset{frameround=fttt, 
frame=trBL, 
stringstyle=\ttfamily,
backgroundcolor=\color{yellow!20},
basicstyle=\footnotesize\ttfamily}
\lstnewenvironment{code}{
\lstset{frame=single,escapeinside=`',
backgroundcolor=\color{yellow!20},
basicstyle=\footnotesize\ttfamily}
}{}



% package to show source code

\definecolor{LightGray}{rgb}{0.9,0.9,0.9}
\newlength{\pixel}\setlength\pixel{0.000714285714\slidewidth}
\setlength{\TPHorizModule}{\slidewidth}
\setlength{\TPVertModule}{\slideheight}
\newcommand\highlight[1]{\fbox{#1}}
\newcommand\icite[1]{{\footnotesize [#1]}}

\newcommand\twotonebox[2]{\fcolorbox{pdcolor2}{pdcolor2}{#1\vphantom{#2}}\fcolorbox{pdcolor2}{white}{#2\vphantom{#1}}}
\newcommand\twotoneboxo[2]{\fcolorbox{pdcolor2}{pdcolor2}{#1}\fcolorbox{pdcolor2}{white}{#2}}
\newcommand\vpspace[1]{\vphantom{\vspace{#1}}}
\newcommand\hpspace[1]{\hphantom{\hspace{#1}}}
\newcommand\COMMENT[1]{}


\newcommand\mx[1]{\begin{math}#1\end{math}}% math expression


\newcommand\placepos[3]{\hbox to\z@{\kern#1
        \raisebox{-#2}[\z@][\z@]{#3}\hss}\ignorespaces} 


%\newtheorem{thm}{Theorem}[chapter] % reset theorem numbering for each chapter

%\usepackage{amsthm}

\newtheorem*{remark}{Remark}
\newtheorem*{theorem}{Theorem}
\newtheorem*{definition}{Definition}
\newtheorem*{proposition}{Proposition}

%\theoremstyle{plain}
%\newtheorem*{theorem}{Theorem}
%\newtheorem*{proposition}[theorem]{Proposition}
%\newtheorem*{definition}[theorem]{Definition}
%\newtheorem*{corollary}[theorem]{Corollary}
%\newtheorem*{lemma}[theorem]{Lemma}

%\theoremstyle{definition}
 % definition numbers are dependent on theorem numbers


%forest
\usepackage{forest}
\usetikzlibrary{arrows.meta, shapes.geometric, calc, shadows}

\colorlet{mygreen}{green!75!black}
\colorlet{col1in}{red!30}
\colorlet{col1out}{red!40}
\colorlet{col2in}{mygreen!40}
\colorlet{col2out}{mygreen!50}
\colorlet{col3in}{blue!30}
\colorlet{col3out}{blue!40}
\colorlet{col4in}{mygreen!20}
\colorlet{col4out}{mygreen!30}
\colorlet{col5in}{blue!10}
\colorlet{col5out}{blue!20}
\colorlet{col6in}{blue!20}
\colorlet{col6out}{blue!30}
\colorlet{col7out}{orange}
\colorlet{col7in}{orange!50}
\colorlet{col8out}{orange!40}
\colorlet{col8in}{orange!20}
\colorlet{linecol}{blue!60}

\definecolor{mynodecolor}{RGB}{198,191,234}

\tikzstyle{flipbox} = [draw=blue!10, fill=green!20, very thick,
rectangle, rounded corners, inner sep=10pt, inner ysep=20pt]
\tikzstyle{fancytitle} =[fill=red, text=white]%, ellipse]
%

%\usepackage{pgfplots}
%\usepackage{tikz}
%%
\usetikzlibrary{positioning, arrows.meta}

\makeatletter
\newcommand\notsotiny{\@setfontsize\notsotiny{7}{8}}
%\newcommand\notsotiny{\@setfontsize\notsotiny\@vipt\@viipt}
\makeatother